\makeglossaries
\newglossaryentry{content}{name=content,description={a file or a directory}}
\newglossaryentry{predicate}{name=predicate,description={an expression
  evaluating to either true or false}}
\newglossaryentry{tree}{name=tree,description={a connected, acyclic graph}}
\newglossaryentry{homoiconicity}{name=homoiconicity,description={a property
  of programs' primary representation being a data structure in their
  language's primitive type}}
\newglossaryentry{closure}{name=closure,description={a function having access
  to variables in the scope it was created in}}

\newacronym{usth}{USTH}{University of Science and Technology of Hanoi}
\newacronym{api}{API}{application programming interface}
\newacronym{id}{ID}{identifier}
\newacronym{cid}{CID}{\gls{content} identifier}
\newacronym{os}{OS}{operating system}
\newacronym{gnu}{GNU}{\acrshort{gnu}'s not Unix}
\newacronym{db}{DB}{database}
\newacronym{dbms}{DBMS}{database management system}
\newacronym{fs}{FS}{file system}
\newacronym{dfs}{DFS}{distributed file system}
\newacronym{io}{I/O}{input/output}
\newacronym{ict}{ICT}{information and communication technology}
\newacronym{jvm}{JVM}{Java virtual machine}
\newacronym{grpc}{gRPC}{Google Remote Procedure Calls}
\newacronym{http}{HTTP}{Hypertext Transfer Protocol}
\newacronym{hdfs}{HDFS}{Hadoop Distributed File System}
\newacronym{ipfs}{IPFS}{InterPlanetary File System}
\newacronym{dag}{DAG}{directed acyclic graphs}
\newacronym{crdt}{CRDT}{conflict-free replicated data type}
\newacronym{jdbc}{JDBC}{Java Database Connectivity}
\newacronym{json}{JSON}{JavaScript Object Notation}
\newacronym{csv}{CSV}{comma-separated values}
\newacronym{xml}{XML}{Extensible Markup Language}
\newacronym{mime}{MIME}{Multipurpose Internet Mail Extensions}
\newacronym{ast}{AST}{abstract syntax \gls{tree}}
\newacronym{qast}{QAST}{query abstract syntax \gls{tree}}
\newacronym{sql}{SQL}{structured query language}
\newacronym{ci}{CI}{continuous integration}
\newacronym{cpu}{CPU}{central processing unit}
\newacronym{repl}{REPL}{read–eval–print loop}

\newcommand{\bit}{b}
\newcommand{\byte}{B}
