\chapter{Evaluation}

\section{Results}
Quantity-wise, the internship has delivered
all expected outcomes and functional requirements.  In summary,
requirement analysis, architecture, design and implementation
of ComLake core were completed.  The \gls{api} was sufficiently
documented\footnote{\url{https://comlake.github.io/comlake.core/api.html}}
for other components to integrate with.

At the end of the internship period, the codebase had a modest size of under
1200 lines of code (excluding all tests and documentation), over a half which
was in Java and the remaining was in Clojure.  The test suite on the other hand
was written entirely in Clojure.  It contained 28 assertions in total
and covered over \SI{96}{\percent} \emph{of the Clojure code}.

Since the first release, eight tags have been published in a course
of two months, alternating between adding new features and delivering
bug fixes.  Despite intensive testing, we were still lacking the confidence
to bump to a non-zero major version number and warrant a stable interface:
the core \gls{api} changed still slightly every other commit.

Speaking of performance, the number of connection a core instance could serve
has been in the order of thousands since the transitioning from RethinkDB
to PostgreSQL, as shown in table~\ref{rdbpsql}.  Note that the drop in
the base (control) operation in comlake.core 0.2.0 is likely due to
the introduction of logging.
\begin{table}\centering
  \caption{Performance in requests per second between comlake.core 0.1.1
  and 0.2.0 using RethinkDB and PostgreSQL as \gls{db} back-end respectively}
  \begin{tabular}{l r r}
    \toprule
    Operation & RethinkDB back-end & PostgreSQL back-end\\
    \midrule
    File upload \emph{and} \gls{db} insert & 325.40 & 357.28\\
    \gls{db} look-up & 121.01 & 5575.89\\
    Download file & 8217.39 & 6238.30\\
    No-operation (control) & 80994.41 & 29788.50\\
    \bottomrule
  \end{tabular}
  \label{rdbpsql}
\end{table}

\section{Discussion}
% clojure op table (data-oriented)
% conn pool importance
% polyglot, note on coverage measure
