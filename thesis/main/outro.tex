\chapter{Conclusion}
Through the development of the presented work, we were able to confirm
certain commonly accepted principles such as the importance of connection pools
to performance with database and the benefits of software freedom.  We also
had a chance to take a deeper look into the trade-offs of using Java and Clojure
together in certain contexts, as well as a concept of query polymorphism.
We believe that the ComLake core service was not only functional and in
some ways useful for the early usage as part of a data lake for \gls{usth},
but also bought some novelty to the data management space, even if minor.

However, it is important to acknowledge that neither the design
nor the implementation is complete or perfect in any criteria.  Performance
around \gls{ipfs} has yet to reach its full potential; a connection pool
and the use of an \gls{ipfs} may drastically improve the relevant benchmark.
Furthermore, metadata extraction is still lacking support for binary
and unstructured data, plus \gls{xml} could be integrated in the future.
A caching invalidation mechanism for those metadata will also be necessary
to stabilizes memory usage in long-running instances.  In addition,
more work can be put into the design of \gls{qast} to further generalize it
and allow more powerful queries.
