\chapter{Introduction}
\section{Motivation}
Researchers and students at \gls{usth} operate with data on a regular basis
and often a dataset is studied by multiple people from different departments
and points in time.  Currently the data are organized manually, usually in
the laboratories' storages, which is prone to duplication and makes
data discovery difficult.

A data lake shared across the university will not only save resources
but also improve productivity and promote interdisciplinary collaborations.
With \gls{usth}'s goal of growing to be an excellent research university
in {\selectlanguage{vietnamese}Việt Nam} and the region~\cite{usth},
building such system is an essential task.

\section{Background}
A \emph{data lake} is a massive repository of multiple types of data
in their raw format at scale for a low cost~\cite{lake}.
The data's schema (structure) is defined on read to minimize data modeling
and integration costs~\cite{lake}.

There has been several technologies exploring these concepts such as
Qri and Kylo, although unsuitable for \gls{usth} due to their goal
and scope differences.  Qri is a peer-to-peer application which may have
unpredictable performance for less popular datasets, and only supports
tabular data~\cite{qri}.  Kylo also does not support binary data,
while requires substantial resources for each instance~\cite{kylo}.

For scaling ease and efficiency, a microservice architecture
could be a good choice.  By arranging the data lake as a collection
of loosely-coupled services, it becomes possible to scale individual ones
individually~\cite{micro}.  In this architecture, the \emph{core} microservice
is defined as the innermost component, communicating directly with the storages
and providing an \gls{api} for others to upload, query and extract data.

\emph{Append-only storages} only allow new data to be appended, whilst ensure
the immutability of existing data.  As immutable data are thread-safe,
they reduce the complexity of the concurrency model, making it easier
to comprehend and reason about~\cite{pure}.  This is particularly useful
in large distributed systems with multiple moving parts.

An immutable \emph{\gls{content}} can be given an \gls{id}, i.e.~a \gls{cid}.
For end-users, we also introduce a higher level concept: \emph{dataset},
composing of not only the content but also relevant metadata for indexing.
Datasets can also be immutable with changes written as new revisions.

Append-only storages' operations boil down to two kinds: appending and reading.
For the latter, sometimes the data are not wanted in their entirety,
but filtered and accumulated.  While data of different types usually requires
different tools and libraries to query upon, it can be possible to use only one
query language for all data and metadata.
In this thesis, such usage is referred to as \emph{query polymorphism}.

\section{Objectives}
The work presented here was done as part of a three-month internship
in collaboration with several other students at \gls{usth}
ICTLab\footnote{\url{https://ictlab.usth.edu.vn}} to build a data lake
for a better management of the university's data.  The internship focused on
the lake's core microservice, which abstracts underlying persistent layers
and perform relevant metadata transformation and discovery.  It should provide
an \emph{internal} interface for data ingestion, (primitive) query and
extraction, while enhancing the discoverability and usability of the datasets.

After the internship period, the resulting codebase shall be maintained
by ICTLab and future students, so the work must be designed, implemented
and documented in a way that ensures such possibility.

\section{Expected Outcomes}
The intended deliverables of the three-month internship are listed as follows:
\begin{itemize}
  \item Requirement analysis of the data lake core
  \item Data lake core's architecture and design
  \item Core \gls{api} design and specification
  \item Implementation and integration with other components
\end{itemize}
